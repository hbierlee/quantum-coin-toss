\documentclass{beamer}
\usetheme{CambridgeUS}
\usepackage[utf8]{inputenc}
\usepackage{listings}
\usepackage{tikz}
\usepackage{graphicx}
\usepackage{blkarray}
\usepackage{xcolor}
\usepackage[normalem]{ulem}
\usepackage{verbatim}

\graphicspath{ {res/} } 
 
%Information to be included in the title page:
\title{Alice (and Bob) in Wonderland}
\author{H. Bierlee \and N. Kankava}
\institute{Uppsala University}
\date{$\pi$-day 2019}
 
% TODO footer is clipping

\definecolor{mygreen}{rgb}{0,0.6,0}
\definecolor{mygray}{rgb}{0.5,0.5,0.5}
\definecolor{mymauve}{rgb}{0.58,0,0.82}

\definecolor{regiongreen}{RGB}{134, 196, 134}
\definecolor{regionpurple}{RGB}{194, 134, 196}
\definecolor{regionorange}{RGB}{255, 150, 0}
\definecolor{regionblue}{RGB}{0, 204, 255}

\lstset{
  backgroundcolor=\color{white},   % choose the background color; you must add \usepackage{color} or \usepackage{xcolor}; should come as last argument
  basicstyle=\footnotesize,        % the size of the fonts that are used for the code
  breakatwhitespace=false,         % sets if automatic breaks should only happen at whitespace
  breaklines=true,                 % sets automatic line breaking
  commentstyle=\color{mygreen},    % comment style
  deletekeywords={...},            % if you want to delete keywords from the given language
  escapeinside={\%*}{*)},          % if you want to add LaTeX within your code
  extendedchars=true,              % lets you use non-ASCII characters; for 8-bits encodings only, does not work with UTF-8
  frame=single,	                   % adds a frame around the code
  columns=flexible,
  keepspaces=true,                 % keeps spaces in text, useful for keeping indentation of code (possibly needs columns=flexible)
  keywordstyle=\color{blue},       % keyword style
  language=C++,                 % the language of the code
  numbers=none,                    % where to put the line-numbers; possible values are (none, left, right)
  rulecolor=\color{black},         % if not set, the frame-color may be changed on line-breaks within not-black text (e.g. comments (green here))
  showspaces=false,                % show spaces everywhere adding particular underscores; it overrides 'showstringspaces'
  showstringspaces=false,          % underline spaces within strings only
  showtabs=false,                  % show tabs within strings adding particular underscores
  stepnumber=2,                    % the step between two line-numbers. If it's 1, each line will be numbered
  stringstyle=\color{mymauve},     % string literal style
  tabsize=2,	                   % sets default tabsize to 2 spaces
  %title=\lstname                   % show the filename of files included with \lstinputlisting; also try caption instead of title
}

 
\begin{document}
 
\frame{\titlepage}
 
\begin{frame}{Problem statement}

\begin{figure}
    \centering
    \includegraphics{blum-abstract}
    \caption{From \emph{COIN FLIPPING BY TELEPHONE: A PROTOCOL FOR SOLVING IMPOSSIBLE PROBLEMS} by M. Blum (1981)}
    \label{fig:blum-abstract}
\end{figure}

\end{frame}

\begin{frame}{A classical solution: lock-box analogy}

\begin{enumerate}
    \item Alice writes down her call (heads/tails) on paper and locks it in a box
    \item Alice sends the box (but not the key) to Bob
    \item Bob tosses the coin and reports the outcome to Alice
    \item Alice reveals who won and sends her key to Bob so he can verify Alice's claim
\end{enumerate}

Relies on the unsure assumption that Bob cannot open the box, in other words: \emph{computational hardness}.

\end{frame}

\begin{frame}{Presentation overview}

\emph{Quantum cryptography:
Public key distribution and coin tossing}\\
\quad by Charles H. Bennett and G. Brassard (1984)
\vfill

\begin{itemize}
    \item \sout{Problem statement}
    \vfill
    \item Recap: behaviour of polarised photons
    \vfill
    \item Remote quantum coin-toss protocol
    \vfill
    \item Drawbacks
    \vfill
    \item Questions
    \vfill
\end{itemize}
\vfill

\end{frame}

\begin{frame}{Recap: behaviour of polarised photons}
\begin{figure}
    \centering
    \includegraphics[width=\textwidth, keepaspectratio]{presentation/res/photon-behaviour.png}
    \caption{From the lecture slides (Feb. 27, 2019)
    % When a photon is measured in the way it was polarised, we will know its original orientation (and the original bit value). If it is measured with the other filter, we get a random bit value.
    }
    \label{fig:my_label}
\end{figure}


\end{frame}

\begin{frame}{Quantum Coin-Toss Protocol}

\begin{figure}
    \centering
    \includegraphics[width=\textwidth, keepaspectratio]{presentation/res/coin-table.pdf}
    \caption{Table from \emph{Quantum cryptography: Public key distribution and coin tossing}}
    \label{fig:my_label}
\end{figure}

\only<1>{1. Alice encodes a random bit-string with a randomly chosen basis and sends them to Bob}
\only<2>{2. Bob randomly chooses a basis for each photon and fills corresponding measurement table and makes a guess}
\only<3>{3. Alice reveals the correct basis and sends original bit-string over classic communication channel}
\only<4>{4. Bob verifies that Alice didn't cheat by comparing bit-string to corresponding measurement table}
\end{frame}

\begin{frame}[plain,c]
    \begin{center}
        \Huge \emph{And now: ..volunteers?}
    \end{center}
\end{frame}


\begin{frame}{Drawbacks}
\begin{itemize}
    \item We need \emph{good} quantum hardware..
        \begin{itemize}
            \item Otherwise Alice might accidentally send multiple photons, allowing Bob to measure multiple times and guess the filter with $>50\%$ probability
        \end{itemize}
    \item ..but we don't want \emph{perfect} quantum hardware.
        \begin{itemize}
            \item Otherwise Alice could cheat by `entangling' photons with the so-called Einstein-Podolsky-Rosen effect
            \item In practice likely impossible to achieve
        \end{itemize}
\end{itemize}
First experimental quantum coin-toss already performed at the Laboratory for Communication and Processing of Information (LTCI) in Paris!
\end{frame}


\begin{frame}{Conclusion}
\begin{itemize}
    \item By the properties of quantum mechanics, it is impossible to cheat:
    \begin{itemize}
        \item Alice cannot cheat, because she can predict only one table with 100\% accuracy, but not the other
        \item Bob cannot cheat, because the photons give no information as to the basis he has to guess
    \end{itemize}
    \vfill
    \item So we rely on the \emph{laws of physics} instead of on \emph{computational complexity}
    \vfill
\end{itemize}
\end{frame}

\begin{frame}[plain,c]
%\frametitle{A first slide}

\begin{center}
    \Huge \emph{Questions..?}
\end{center}

\end{frame}


\begin{frame}[plain,c]
%\frametitle{A first slide}

\begin{center}
    \Huge \emph{Thank you for listening!}
    
\end{center}
\vfill
    \large
    Report, slides and demo source code available at:\\ \quad \url{https://github.com/hbierlee/quantum-coin-toss}
    
\end{frame}

\end{document}


    % \begin{itemize}
    %     \item \begin{verbatim}
    %          0 1 0 1 0 0 
    %         \end{verbatim}
    %     % \item \begin{verbatim}
    %     %         RECTILINEAR
    %     %     \end{verbatim}
    % \end{itemize}